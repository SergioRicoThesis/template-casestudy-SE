%% 
%% Copyright 2007-2020 Elsevier Ltd
%% 
%% This file is part of the 'Elsarticle Bundle'.
%% ---------------------------------------------
%% 
%% It may be distributed under the conditions of the LaTeX Project Public
%% License, either version 1.2 of this license or (at your option) any
%% later version.  The latest version of this license is in
%%    http://www.latex-project.org/lppl.txt
%% and version 1.2 or later is part of all distributions of LaTeX
%% version 1999/12/01 or later.
%% 
%% The list of all files belonging to the 'Elsarticle Bundle' is
%% given in the file `manifest.txt'.
%% 

%% Template article for Elsevier's document class `elsarticle'
%% with numbered style bibliographic references
%% SP 2008/03/01
%%
%% 
%%
%% $Id: elsarticle-template-num.tex 190 2020-11-23 11:12:32Z rishi $
%%
%%
\documentclass[preprint,12pt]{elsarticle}
\usepackage[most]{tcolorbox}

%% Use the option review to obtain double line spacing
%% \documentclass[authoryear,preprint,review,12pt]{elsarticle}

%% Use the options 1p,twocolumn; 3p; 3p,twocolumn; 5p; or 5p,twocolumn
%% for a journal layout:
%% \documentclass[final,1p,times]{elsarticle}
%% \documentclass[final,1p,times,twocolumn]{elsarticle}
%% \documentclass[final,3p,times]{elsarticle}
%% \documentclass[final,3p,times,twocolumn]{elsarticle}
%% \documentclass[final,5p,times]{elsarticle}
%% \documentclass[final,5p,times,twocolumn]{elsarticle}

%% For including figures, graphicx.sty has been loaded in
%% elsarticle.cls. If you prefer to use the old commands
%% please give \usepackage{epsfig}

%% The amssymb package provides various useful mathematical symbols
\usepackage{amssymb}
%% The amsthm package provides extended theorem environments
%% \usepackage{amsthm}

%% The lineno packages adds line numbers. Start line numbering with
%% \begin{linenumbers}, end it with \end{linenumbers}. Or switch it on
%% for the whole article with \linenumbers.
%% \usepackage{lineno}

\journal{Template Case Study in Software Engineering}

\begin{document}

\begin{frontmatter}

%% Title, authors and addresses

%% use the tnoteref command within \title for footnotes;
%% use the tnotetext command for theassociated footnote;
%% use the fnref command within \author or \address for footnotes;
%% use the fntext command for theassociated footnote;
%% use the corref command within \author for corresponding author footnotes;
%% use the cortext command for theassociated footnote;
%% use the ead command for the email address,
%% and the form \ead[url] for the home page:
%% \title{Title\tnoteref{label1}}
%% \tnotetext[label1]{}
%% \author{Name\corref{cor1}\fnref{label2}}
%% \ead{email address}
%% \ead[url]{home page}
%% \fntext[label2]{}
%% \cortext[cor1]{}
%% \affiliation{organization={},
%%             addressline={},
%%             city={},
%%             postcode={},
%%             state={},
%%             country={}}
%% \fntext[label3]{}

\title{Template for a Case Study Protocol in Software Engineering}

%% use optional labels to link authors explicitly to addresses:
%% \author[label1,label2]{}
%% \affiliation[label1]{organization={},
%%             addressline={},
%%             city={},
%%             postcode={},
%%             state={},
%%             country={}}
%%
%% \affiliation[label2]{organization={},
%%             addressline={},
%%             city={},
%%             postcode={},
%%             state={},
%%             country={}}

\author[inst1]{Author One}

\affiliation[inst1]{organization={Department One},%Department and Organization
            addressline={Address One}, 
            city={City One},
            postcode={00000}, 
            state={State One},
            country={Country One}}

\author[inst2]{Author Two}
\author[inst1,inst2]{Author Three}

\affiliation[inst2]{organization={Department Two},%Department and Organization
            addressline={Address Two}, 
            city={City Two},
            postcode={22222}, 
            state={State Two},
            country={Country Two}}

\begin{abstract}
%% Text of abstract
\textbf{Background and Context:} What are the theoretical aspects that the case will be about, or in what context the case study will take place? \\
\textbf{Objective:} What is the primary goal of the case study?.\\
\textbf{Methodology:} What type of case study is e.g., exploratory, evaluatory?. What are data collection actions planned?\\
\textbf{Expected results:} What are the expectations regarding the outcomes of the study?
\end{abstract}

%%Graphical abstract
%\begin{graphicalabstract}
%\includegraphics{grabs}
%\end{graphicalabstract}

%%Research highlights
%\begin{highlights}
%\item Research highlight 1
%\item Research highlight 2
%\end{highlights}

\begin{keyword}
%% keywords here, in the form: keyword \sep keyword
case study \sep software engineering \sep template
%% PACS codes here, in the form: \PACS code \sep code
%\PACS 0000 \sep 1111
%% MSC codes here, in the form: \MSC code \sep code
%% or \MSC[2008] code \sep code (2000 is the default)
%\MSC 0000 \sep 1111
\end{keyword}

\end{frontmatter}

%% \linenumbers

%% main text
\section*{About this template}
Case study research is a well-established technique for investigating different phenomena in software engineering. This template aims to provide a squeleton for designing a Case Study. 
The template is mainly based on two resources. One is the facto reference for conducting case studies in software engineering by Runeson, and Höst 2009 \textit{\textbf{Guidelines for conducting and reporting case study research in software engineering}}\cite{runeson2009guidelines}. The other is the book \textit{\textbf{Case Study Research in software engineering: Guidelines and examples}}\cite{runeson2012case}.

I am preparing this template since every time I was planning a new study, I felt like starting from scratch and lacking a template to add content to design the study. Although there are checklists of aspects to consider in case study research, I have not seen a template to start designing a case study. Additionally, I also provide some hints and pointers to helpful resources. 


\section{Background}
\label{sec:background}


\subsection{Related work}
\label{subsec:relatedwork}
Identify previous researth on the topic.

\subsection{Research questions}
\label{subsec:rqs}
Define the research questions for this study. 

\section{Design}
\label{sec:design}
Single case vs multiple case. How the case is connected to the research questions?

What is the object of study, units of analysis ?

How the research questions will be addressed? Add propositions or sub-research questions to the research questions. 

\section{Selection}
\label{sec:selection}
Criteria for case selection

\section{Roles and procedures}
\label{sec:roles}
Who is involved in the case study?

What are the roles and responsibilities of the researchers?


\section{Data collection}
\label{sec:datacollection}
What is the data that will be collected?

Define a data collection plan.

How the data will be stored?

\section{Analysis}
\label{sec:analysis}
What id the criteria for interprete the casy study findings?

How the data findings will answer the research questions?

\section{Plan validity}
\label{sec:planvalidity}

\subsection{General validity}
\label{subsec:general}
Check against the following checklist:

\begin{tcolorbox}[colback=black!5!white,colframe=black!75!white,title=Checklist for Case Design from \cite{runeson2009guidelines}]
    \begin{enumerate}
      \item What is the case nad its units of analysis?
      \item Are clear objectives, preliminary research questions, hypotheses (if any) defined in advance?
      \item Is the theoretical basis- relation to existing literature or other cases -defined?
      \item Are the authors intentions with the research made clear?
      \item Is the case adequately defined (size, domain, process, subjects, etc.)?
      \item Is a cause-effect relation under study? If yes, is it possible to distinguish the cause from other factos using the proposed design?
      \item Does the design invovle data from multiple sources (data triangulation), using multiple methods (method triangulation)?
      \item Is there a rationale behind the selection of subjects, roles, artifacts, viewpoints, etc.?
      \item Is the specified case relevant to validity address the research questions (construct validity)?
      \item Is the integrity og individual/organizations taken into account?
    \end{enumerate}
  \end{tcolorbox}

\subsection{Construct validity}
\label{subsec:construct}
Show that the correct operational meausures are planned for the concepts in the study. Some tactics to consider: multiple source of evidence, establish chains of evidence, expert reviews. 

\subsection{External validity}
\label{subsec:external}
Identify the domain to which study is relevant. Use multiple-case studies to investigate outcomes in different contexts. 

\section{Study limitations}
\label{sec:limitations}
Identify potential conflicts of interest. 

\section{Reporting}
\label{sec:reporting}
How the results will be reported? 

Venues for reporting. Specify the reporting format.

\section{Schedule}
\label{sec:schedule}
Time plan for planning, data collection, data analysis, and reporting.


 %tex file with the case design

%% The Appendices part is started with the command \appendix;
%% appendix sections are then done as normal sections
\appendix
\section{Checklist for data collection}
\begin{tcolorbox}[colback=black!5!white,colframe=black!75!white,title=Checklist for Preparation for data collection \cite{runeson2009guidelines}]
\begin{enumerate}
    \setcounter{enumi}{10}
    \item Is a case study protocol for data collection and analysis derived (what, why,
    how, when)? Are procedures for its update defined?
    \item Are multiple data sources and collection methods planned (triangulation)?
    \item Are measurement instruments and procedures well defined (measurement definitions, interview questions)?
    \item Are the planned methods and measurements sufficient to fulfill the objective of the study?
    \item Is the study design approved by a review board, and has informed consent obtained from individuals and organizations?
    \item Is data collected according to the case study protocol?
    \item Is the observed phenomenon correctly implemented (e.g., to what extent is a design method under study actually used)?
    \item Is data recorded to enable further analysis?
    \item Are sensitive results identified (for individuals, the organization or the project)?
    \item Are the data collection procedures well traceable?
    \item Does the collected data provide ability to address the research question?
\end{enumerate}
\end{tcolorbox}

\section{Checklist for analysis of collected data}
\begin{tcolorbox}[colback=black!5!white,colframe=black!75!white,title=Checklist for Preparation for data collection \cite{runeson2009guidelines}]
\begin{enumerate}
    \setcounter{enumi}{21}
    \item Is the analysis methodology defined, including roles and review procedures?
    \item Is a chain of evidence shown with traceable inferences from data to research
    questions and existing theory?
    \item Are alternative perspectives and explanations used in the analysis?
    \item Is a cause–effect relation under study? If yes, is it possible to distinguish the
    cause from other factors in the analysis?
    \item Are there clear conclusions from the analysis, including recommendations for
    practice/further research?
    \item Are threats to the validity analyzed systematically and countermeasures taken? (Construct, internal, external, reliability)
\end{enumerate}
\end{tcolorbox}

\section{Checklist for reporting case studies}
\begin{tcolorbox}[colback=black!5!white,colframe=black!75!white,title=Checklist for Preparation for data collection \cite{runeson2009guidelines}]
\begin{enumerate}
    \setcounter{enumi}{27}
    \item  Are the case and its units of analysis adequately presented?
    \item  Are the objective, the research questions and corresponding answers reported?
    \item  Are related theory and hypotheses clearly reported?
    \item Are the data collection procedures presented, with relevant motivation?
    \item Is sufficient raw data presented (e.g., real-life examples, quotations)?
    \item Are the analysis procedures clearly reported?
    \item Are threats to validity analyses reported along with countermeasures taken to reduce threats?
    \item Are ethical issues reported openly (personal intentions, integrity issues, confidentiality)
    \item Does the report contain conclusions, implications for practice, and future research?
    \item Does the report give a realistic and credible impression?
    \item Is the report suitable for its audience, easy to read, and well structured?
\end{enumerate}
\end{tcolorbox} %tex file with the appendix
%\section{Sample Appendix Section}
%\label{sec:sample:appendix}
%Lorem ipsum dolor sit amet, consectetur adipiscing elit, sed do eiusmod tempor section \ref{sec:sample1} incididunt ut labore et dolore magna aliqua. Ut enim ad minim veniam, quis nostrud exercitation ullamco laboris nisi ut aliquip ex ea commodo consequat. Duis aute irure dolor in reprehenderit in voluptate velit esse cillum dolore eu fugiat nulla pariatur. Excepteur sint occaecat cupidatat non proident, sunt in culpa qui officia deserunt mollit anim id est laborum.

%% If you have bibdatabase file and want bibtex to generate the
%% bibitems, please use
%%
 \bibliographystyle{elsarticle-num} 
 \bibliography{references}

%% else use the following coding to input the bibitems directly in the
%% TeX file.

% \begin{thebibliography}{00}

% %% \bibitem{label}
% %% Text of bibliographic item

% \bibitem{}

% \end{thebibliography}
\end{document}
\endinput
%%
%% End of file `elsarticle-template-num.tex'.
